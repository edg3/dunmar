\documentclass[10pt,twoside,twocolumn]{article}

\usepackage[bg-none]{dnd/dnd}	%options: bg-a4, bg-letter, bg-full, bg-print; default are bg-letter and bg-full
\usepackage[utf8]{inputenc}

\setcounter{secnumdepth}{5}
\setcounter{tocdepth}{5}

\title{Dunmar: Curse Caverns}
\author{Ernest "edg3" Loveland}

% Start document
\begin{document}
\fontfamily{ppl}\selectfont % Set text font

% Content goes after this
\maketitle
\tableofcontents

% quotebox => descriptive; paperbox => rules/mechanics/examples; commentbox => additional information

\section{About}
This one-off is built in the Dunmar setting (see: Dunmar DM Guide) and is of high difficulty for 3 to 6 level 5 characters. It is recommended you play with the optional mechanic "Insanity". \\

There are two ways you can run this, a simple mode or an extensive mode. The extensive mode would require multiple levels of dungeon and a few days worth of adventuring (4 or more sessions of 3 hours) whereas the simple mode is geared towards a single session. Options marked "[Extend]" are additional place and happenings you can introduce.

\subsection{Optional Rule: Insanity}

\section{Chapter 1: Introduction}
\begin{quotebox}
As you move tiredly into Shalk, a city on the lower crust of Rengar you are awestruck by the sheer magnitude and beauty. The whole city built from shards of the rock-hard ice all around reflecting lights of differing colour all around. As you cross the threshold, not marked but clearly an obvious border the bustling city opens up below you cut into the ice. \\

A dragonborn in light armor near you rushes up, "sirs, madams, your attention! The king, his lord Vesti Dragon of Shalk has put a bounty on a magical item and will reward adventuring parties that would bring it back for him." As he proclaims this he hurriedly shoves carefully wrapped scrolls into each person that arrived with's hands, each tied by a quaint blue ribbon. \\

"Time is short, the meeting is a few minutes from now at the Bottlesworth Bar,| he hurriedly shouts the directions to the bar and heads off.\\

Opening the scroll in ornate lettering: The king will award the bearer of this document 10000gp for the successful retrieving and handing in of a magical item of his interest. More information at the Bottlesworth Bar at noon on the eve of the new moon. \\

The promise of adventure and riches is definitely enough to pique the interest of many and after a short trip you make your way to the bar. A gentleman takes the stage as you start filling up the last seat, clears his throat and begins to talk "in 3 hours time a magical portal to a set of hidden caverns North of the city will open, this happens once every 20 years and we have predicted it will be open today, so as you can tell time is short and being hasty is important. These caverns are only open for 3 days and are extremely dangerous, if you are no longer interested, this is your last chance to leave..." \\

He glances around the room and is greeted with a deathly silence, "good" he continues. A surly man at the door closes and bars it from the inside. "We have a request from the king himself - he is interested in a hidden works that was lost in the Curse Caverns - the book contains information key to him ascending to a higher plane and thus he wants it." \\

"The compensation as you know is very worthy of the deeds it will take to retrieve it but the caverns are not in our domain of control and are full of traps. We have a hidden pathway out of here that leads to entrances to the caverns which you can use - we need to be discrete as other groups are seeking the writings themselves."
\end{quotebox}

At this point your adventurers would need to do any last minute preparation, you can offer them whatever they need if they don't already have it, otherwise they will be ushered into a secret tunnel under the inn that leads out of the pub under the ice of the city. \\

\subsection{[Extend] Traveling to the Entrance}
TODO \\

\section{Chapter 2: Entering the Caverns}
You come out into the open facing a solid stone wall that towers above you leading up to Upcrust. Another party just ahead of you gets lead up to the wall of stone and they walk into the wall as if through air and dissapear.

\begin{quotebox}
"Alright adventurers, you are next. We have no idea where you will end up in there, it is up to you guys to find the requested item and make your way out. Remember that if you are still there in 3 days you are likely to be sealed in there for the next 20 years!" the guard gestures forward towards the wall, handing you a rudimentary map, "You'll need this!"
\end{quotebox}

The caverns can follow any layout, it is recommended you keep track for yourself with a number of other adventuring "parties" that are wanting to claim the same prize.

\subsection{Traveling the Caverns}
When a party moves around a corner roll from this table to decide what they see in the corridor. Roll from this table to discern what all they face, note that not everything is a combat encounter and you will need to handle this appropriately.

\begin{dndtable}

\end{dndtable}

This is just as much about mystery as it is about finding the object, and more so it is about the horror. \\

\begin{commentbox}
In the Curse of Strahd campaign notes it suggests 2 main things to keep in mind: \\

Describe things in horrific detail, the smells, the decay, the dark dinghy colours. Seeing horrific things leads to a sanity check \\

When something is pretty (a flower, a pedestal, an item) it should be eye catching, describe it to the players in all its beauty to contrast the darkness. On seeing something that isn't horrific you can give the players back a little bit of sanity.
\end{commentbox}

The goal of your players will be to retrieve the amulet, survive ambushes by other parties and get out alive and sane without permanent sanity problems

\end{document}